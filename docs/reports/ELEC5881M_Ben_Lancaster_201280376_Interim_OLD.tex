\documentclass[11pt,a4paper]{report}

% font
%\usepackage{fontspec}
%\setmainfont{Times New Roman}
%\usepackage{tgschola}

%\usepackage{helvet}
%\renewcommand{\familydefault}{\sfdefault}

\usepackage[libertine,cmintegrals,cmbraces,vvarbb]{newtxmath}

\linespread{1.3}

%\usepackage[margin=0.8in]{geometry}
\usepackage[utf8]{inputenc}
\usepackage{amsmath}
\usepackage{amsfonts}
\usepackage{amssymb}

%https://www.overleaf.com/help/41-can-i-add-inline-or-margin-comments-to-the-pdf
\usepackage{xargs} % Use more than one optional parameter in a new comma
\usepackage[colorinlistoftodos,prependcaption,textsize=small]{todonotes}
\newcommandx{\unsure}[2][1=]{\todo[linecolor=red,backgroundcolor=red!25,bordercolor=red,#1]{#2}}
\newcommandx{\change}[2][1=]{\todo[linecolor=blue,backgroundcolor=blue!25,bordercolor=blue,#1]{#2}}
\newcommandx{\info}[2][1=]{\todo[linecolor=green,backgroundcolor=green!25,bordercolor=green,#1]{#2}}
\newcommandx{\improvement}[2][1=]{\todo[linecolor=purple,backgroundcolor=purple!25,bordercolor=purple,#1]{#2}}
\newcommandx{\thiswillnotshow}[2][1=]{\todo[disable,#1]{#2}}


%https://tex.stackexchange.com/questions/3001/list-sections-of-chapter-at-beginning-of-that-chapter
% !!! NEEDS TO BE ABOVE HYPEREF !!!
\usepackage{titletoc}

% https://www.sharelatex.com/learn/Hyperlinks
\usepackage{hyperref}
\hypersetup{
    colorlinks,
    %citecolor=gray,
    citecolor=blue,
    filecolor=black,
    linkcolor=blue,
    urlcolor=blue,
    linktoc=all
}
\usepackage{float}

% https://tex.stackexchange.com/questions/94224/how-to-create-a-list-with-a-fixed-prefix-and-incremental-numbers
\usepackage{enumitem}

\usepackage{lipsum}% http://ctan.org/pkg/lipsum

%% Bibliography/references packages
%\usepackage[comma]{natbib}
\usepackage[square,sort,comma,numbers]{natbib}
%\bibliographystyle{agsm}
%\bibliographystyle{dcu}
\bibliographystyle{unsrt}

% CODE
%\usepackage{listings}
%\usepackage{xparse}
%\usepackage{verbatim}
%\NewDocumentCommand{\codeword}{v}{%
%\texttt{\textcolor{blue}{#1}}%
%}
%s\lstset{language=C,keywordstyle={\bfseries \color{blue}}}
\usepackage[outputdir=build]{minted}
%https://www.sharelatex.com/learn/Code_Highlighting_with_minted#Reference_guide
\usemintedstyle{vs} % HAS ERROR BOXES!
%\usemintedstyle{pastie}
%\usemintedstyle{manni}

%% https://en.wikibooks.org/wiki/LaTeX/List_Structures
\usepackage{scrextend}

% tables, row colour
\usepackage{tabularx,colortbl}
% For vertical centering text in X column
\renewcommand\tabularxcolumn[1]{m{#1}}

% https://tex.stackexchange.com/questions/22751/how-to-force-table-caption-on-top
%\usepackage[tableposition=top]{caption}
\usepackage{float}
\floatstyle{plaintop}
\restylefloat{table}

% https://en.wikibooks.org/wiki/LaTeX/List_Structures
\usepackage{enumitem}

% https://jansoehlke.com/2010/06/strikethrough-in-latex/
\usepackage{ulem}

% https://tex.stackexchange.com/questions/337/how-to-change-certain-pages-into-landscape-portrait-mode
\usepackage{pdflscape}

%% Report variables
\newcommand{\scname}{PRCO304}
\newcommand{\dlatestv}{3.14}

\definecolor{babyblueeyes}{rgb}{0.63, 0.79, 0.95}
\definecolor{ballblue}{rgb}{0.13, 0.67, 0.8}
\definecolor{beaublue}{rgb}{0.74, 0.83, 0.9}
\definecolor{blue8}{rgb}{0.09, 0.39, 0.67}
\definecolor{blue9}{rgb}{0.00, 0.30, 0.51}
\definecolor{blue9d}{rgb}{0.00, 0.21, 0.36}


%\usepackage{etoolbox}
%\patchcmd{\chapter}{\thispagestyle{plain}}{\thispagestyle{fancy}}{}{}

%https://tex.stackexchange.com/questions/75667/change-colour-on-chapter-section-headings-lyx
\usepackage{sectsty}
%\chapterfont{\color{blue9d}}
%\sectionfont{\color{blue9d}}
%\subsectionfont{\color{blue9d}}
%\subsubsectionfont{\color{blue9d}}
%\allchapterfont{\itshape}

\usepackage{titlesec}

\usepackage{array,booktabs,arydshln,xcolor}
\usepackage{xcolor}% http://ctan.org/pkg/xcolor
\usepackage{fancyhdr}% http://ctan.org/pkg/fancyhdr
\fancypagestyle{plain}{%
	\renewcommand{\headrulewidth}{3pt}
	\renewcommand{\headrule}{\hbox to\headwidth{%
		\color{blue9}\leaders\hrule height \headrulewidth\hfill}}
	\renewcommand{\footrulewidth}{3pt}
	\renewcommand{\footrule}{\hbox to\headwidth{%
		\color{blue9}\leaders\hrule height \headrulewidth\hfill}}
	
	%\fancyhf{}
	%\fancyhead[LE]{\textbf{\leftmark}}
	%\fancyhead[RE]{\textbf{\scname{}}}
	%\fancyhead[LO]{\textbf{\scname{}}}
	%\fancyhead[RO]{\textbf{\rightmark}}

	%\fancyfoot[LE]{\textbf{\thepage}}
	%\fancyfoot[RE]{\textbf{\scname{} Configuration Guide}}
	%\fancyfoot[LO]{\textbf{\scname{} Configuration Guide}}
	%\fancyfoot[RO]{\textbf{\thepage}}
}

%s comments
\usepackage{verbatim}

%inline graphs
\usepackage{wrapfig}
% multiple figures on line
\usepackage{subfig}

\usepackage{graphicx}
\graphicspath{{img/}}

% Caption font size
% https://tex.stackexchange.com/questions/86120/font-size-of-figure-caption-header
\usepackage[font=scriptsize,labelfont=bf]{caption}

%\setlength{\belowcaptionskip}{-10pt}
%\setlength{\abovecaptionskip}{-5pt} % Chosen fairly arbitrarily


\usepackage{fancyhdr}
\pagestyle{fancy}
\lhead{\rightmark}
\chead{}
%\rhead{FPGA-based RISC Microprocessor and Compiler (Rev. \dlatestv{})}
\rhead{PRCO304 (Rev. \dlatestv{})}
\lfoot{Ben Lancaster 10424877}
\cfoot{}
\rfoot{Page \thepage}

\renewcommand{\subsectionmark}[1]{\markright{\thesubsection\ #1}}


%% Make bibliography show in table of contents
%% https://tex.stackexchange.com/questions/8458/making-the-bibliography-appear-in-the-table-of-contents
\usepackage[nottoc,numbib]{tocbibind}
%% ^^^ overwrites \bibname, so set it back
\renewcommand{\bibname}{References}


\begin{document}
\arrayrulecolor{blue9}

\makeatletter
\DeclareRobustCommand*{\nameref}{%
\color{blue}%
        \@ifstar\T@nameref\T@nameref
        }%
\makeatother

% Real front cover page
\newpage
\begin{titlepage}
\begin{center}

\vspace*{5cm}
\Large

{\color{blue9d}
\textbf{
%%PRCO304 - Project Initiation Document
%Highlight Reports
FPGA-based RISC Microprocessor and Compiler (Rev. \dlatestv{})
}
}

\vspace{0.4cm}
\large
%%Space optimised FPGA-based side-microprocessor.
PRCO304 - Final Stage Computing Project
%%EMBEDDED CPU - FPGA-based RISC microprocessor

\vspace{4cm}
\textbf{Ben Lancaster 10424877}\\
\today 


\end{center}

\end{titlepage}

\pagestyle{plain}

\section*{Revision History}
\begin{table}[h]
\def\arraystretch{1.3}
    \begin{tabularx}{\textwidth}{|l|l|X|}
    \hline
    Date & Version & Changes \\
    \arrayrulecolor{blue9}
	\specialrule{2pt}{-2pt}{0pt}
	20/05/2018 & 3.14 & Add background research to appendix. \\ \hline
	19/05/2018 & 3.13 & Update abstract to align with guidelines. \\ \hline
	19/05/2018 & 3.12 & Fix ISA pseudo-codes. \\ \hline
	15/05/2018 & 3.11 & Correct CMP (\ref{isa:cmp}) instruction bits. \\ \hline
	04/05/2018 & 3.10 & Add section \ref{sect:core_analysis} {\nameref{sect:core_analysis}} to describe performance of the embedded processor. \\ \hline
	04/05/2018 & 3.00 & Add section {\nameref{sect:preface}}. \\ \hline
	03/05/2018 & 2.90 & Add acknowledgements and glossary. \\ \hline
	20/04/2018 & 2.81 & Add details of emulator and it's integration with unit testing.\\ \hline
	30/03/2018 & 2.80 & Add details of Argument Variables implementation. \\ \hline
	30/03/2018 & 2.70 & Add description of compiler function requirements. \\ \hline
	30/03/2018 & 2.60 & Add word count below TOC. \\ \hline
	29/03/2018 & 2.50 & Add chapter table of contents. \\ \hline
	29/03/2018 & 2.40 & Add section \ref{sect:core_test} \nameref{sect:core_test}. \\ \hline
	28/03/2018 & 2.30 & Add section \ref{sect:compiler_cg_vars} \nameref{sect:compiler_cg_vars}. \\ \hline
	24/03/2018 & 2.20 & Add section \ref{sect:compiler_cg_push_pop} \nameref{sect:compiler_cg_push_pop}. \\ \hline
	22/03/2018 & 2.10 & Add section \ref{sect:compiler_ast} \nameref{sect:compiler_ast}. \\ \hline
	15/03/2018 & 2.00 & Add section \ref{sect:compiler_grammar} \nameref{sect:compiler_grammar}. \\ \hline
	11/03/2018 & 1.00 & Initial section outline. \\ \hline
    \end{tabularx}
    \caption{Document revisions.}
\end{table}

\newpage
\chapter*{Acknowledgements}
I would like to thank my project supervisors Nigel Barlow and Serge Thill for their support and guidance throughout this project. 
\\\\
I would also like to thank James Spalding (Spirent Communications) and firmware team for their encouragement, ideas, and industrial sponsorship supporting this final project.

\newpage
\chapter*{Abstract}
\startcontents[chapters]
\printcontents[chapters]{}{1}{}
\noindent
This report describes a software and electronic development project to design, implement, and verify, a new embedded processor and architecture targeting small FPGA devices, and a high-level code compiler for generating executable code for the processor.
\\\\
This report outlines the design decisions of the new embedded processor's instruction set architecture, register sets, and compiler. In addition, a new high-level programming language is introduced that can be compiled into executable code for the embedded processor. Implementation details of the embedded processor and compiler is described, including pipelining, memory-management, code-generation, and optimisations. 
\\\\
A combination of PRINCE2 and Agile methodologies are employed, allowing for incremental development of both core and compiler, yet with risk, quality, and development stages identified. Descriptions of the project management processes and drawings are provided throughout the report.
\\\\
A project post-mortem is performed to reflect on the achievements of the project with respect to the initial project objectives and to discuss further improvements to the project management, design, implementation, and verification, of the project. It was found that the combination of Agile and PRINCE2 methodologies aided the development of the project. The core was fully implemented on Spartan-6 FPGA hardware, with a performance of 10 MIPS and 0.2 IPC at 40MHz, but was found to utilise too many slice resources. The compiler was able to produce optimised machine code but failed to identify some types of common optimisations, resulting in larger program sizes.
\\\\
The processor specification, compiler usage guides, and other project management  documents are provided in the appendices.

\newpage
\renewcommand*\contentsname{Table of Contents}
%TC:ignore
{\hypersetup{linkcolor=black}
\tableofcontents
\listoffigures
\listoftables}
\newpage

\subsection*{Word Count}
Words count: 10600

\subsection*{Sources}
\begin{description}[style=nextline]
\item[GitHub (with git repository metadata)]{\url{https://github.com/bendl/prco304}}
\item[OneDrive (raw files)]{\url{https://liveplymouthac-my.sharepoint.com/:f:/g/personal/ben_lancaster_students_plymouth_ac_uk/EmzpOaOZhyZIvDhGLSUQC8oBbRX5iO6Ujwe81TUa1VlV-A?e=Ndd677}}
\end{description}

\subsection*{Glossary}
\begin{description}
\item[SoC]{System on Chip}
\item[FPGA]{Field-programmable Gate Array}
\item[ASIC]{Application-specific integrated circuit}
\item[RISC]{Reduced instruction set computer}
\item[ISA]{Instruction set architecture}
\item[Pipeline]{The control and transfer of data in a system}
\item[RTL]{Register-transfer level}
\item[HDL]{Hardware description language}
\item[Synthesis]{A process of transforming HDL into RTL}
\item[Assembler]{A process of transforming a higher representation of code into a machine executable format}
\item[AST]{Abstract Syntax Tree}
\item[SSA]{Static Single Assignment}
\item[IR]{Intermediate representation}
\item[Opcode]{Bits of an instruction indicating the type of operation}
\item[Operand]{A parameter or part of an instruction}
\item[Imm8]{8-bit immediate value}
\item[Simm5]{5-bit signed immediate value}
\item[CMake]{A cross-platform project generator}
\end{description}
%TC:endignore

\newpage
\chapter*{Preface}
\label{sect:preface}
This report discusses the design, implementation, and verification, of an FPGA-based embedded processor (\scname{} processor) and compiler.


A field-programmable gate array (FPGA) is a reprogrammable logic device that enables digital electronic designs to be realised and executed on-chip. FPGAs utilise configurable logic blocks (CLB) that can be configured to emulate primitive gates. Connecting these primitive gates with others allows the engineer to realise complex logic such as combinational gates, multiplexers, and lookup-tables. Modern FPGA devices also include components such as block-RAMs, DLLs, and DSP blocks.

An embedded processor will be designed using the Verilog hardware description language (HDL). A HDL can be thought of as a front-end to a software compiler. Other HDL front-ends exist such as VHDL. The HDL is synthesised into an retargetable intermediate form consisting of nets - a form describing the gates, flip-flops, and their interconnections. After this synthesis takes place, the intermediate net list form is transformed into implementation specific forms. As this processor is designed for FPGA implementation, processes such as "place and route" are utilised to map the net list to physical resources on the FPGA. Different processes are performed for different implementations, such as for ASICs and CPLDs.

\newpage
\chapter{Embedded Processors and Compilers}
{\hypersetup{linkcolor=black}
\startcontents[chapters]
\printcontents[chapters]{}{1}{}
}

\section{Introduction}
Modern computing and electronics equipment, like function generators, oscilloscopes, and spectrum analysers, use FPGAs to implement their compute intensive logic. These FPGAs are often accompanied by a small, low-cost, microprocessor to supervise and provide interfaces to external peripherals.

The aim of this project is to implement this side-microprocessor into the FPGA to save on BOM costs, PCB  space,  and  power  costs,  which  contribute  to  higher  development  and  product  costs.  While  savings can  be  made  by  the  lack  of  side  microprocessor,  the  product  may  need  a  larger  FPGA  to  accommodate the embedded microprocessor.  The project will produce a small, soft-core, CPU design and compiler. Although there is no direct client in this project, I believe this project will produce an attractive product for FPGA-based product designers wishing to employ an embedded processor solution as well as improve my knowledge and experience in this field.

This report details the design considerations, implementation, and verification, of a new embedded processor architecture and a high-level code-compiler targeting it.

\section{Background}
Embedded processors are becoming more and more present in many products, ranging from toys to outdoor sensory recording systems to test and measurement tools. Embedded processors are specifically designed for remote and constrained environments, where power consumption, operating temperature, and form factor are extremely constricted. 

FPGAs are powerful devices. They allow virtually any digital electronics design to be programmed post-manufacturing to the device. This makes FPGAs safer and cheaper to implement complex digital logic and are great alternatives to ASICs, which cannot be reprogrammed and have steep initial development costs. FPGAs are being utilised in an increasing number of new areas, such as machine vision for image processing and machine learning for faster learning  \cite{fpgacloud}.
 
\subsection{Soft-core Embedded Processors}
There exists many commercial and open-source embedded processors, each providing different features and specialities such as digital signal processing, analogue components, instruction set architectures, and interfaces.

Research has been performed to identify existing embedded processor features and characteristics. This research has been used to identify requirements and desired functionality of this new processor architecture. This research can be found in appendix \ref{current_impl} {\nameref{current_impl}}.

From this research, it was clear that the new processor was to be a lightweight design with RISC architecture. PicoBlaze, with it's on-chip scratch memory and low resource footprint, is a primary inspiration for this new processor. 

\subsection{Compilers}
Compilers are used to transform high-level code into a lower form. This is particularly useful for large applications where abstracting low-level details can result in faster development and better optimisation. Most modern open-source compilers today have retargetable front-end, optimisation, and back-end components that allow new architectures and grammars to be implemented while still utilising existing compiler features. 

Research into existing compilers has been performed. This research is required to identify a suitable compiler that would allow easy creation of a back-end for the new processor architecture. 

After comparing compilers such as LLVM and 8CC, it was decided to build a new compiler from scratch rather than implement a back-end for an existing compiler. This was to satisfy a chief requirement of the project: to improve my learning and experience in processor architecture, low-level programming, and compilers.

\newpage

\section{Project Overview}
This primary aim of this project is to improve my knowledge and experience in embedded processors, SoC design, computer architecture, and compilers. To do this, I will design an efficient and cost-saving alternative for board and hardware product designers utilising side-microprocessors by designing, implementing, and demonstrating, a small, portable, FPGA processor core design to be used in-place of the side-microprocessor. 

The processor core will implement it's own pipeline and instruction set architecture and so a compiler and assembler will also be provided so that software code can easily be executed on the processor. The new processor core and compiler tool chain will be named \textit{\scname{}}.


%TC:endignore
\end{document}
